\documentclass[10pt]{article}

\usepackage[latin1]{inputenc}
\usepackage[T1]{fontenc}
\usepackage[francais]{babel}
\usepackage{soul}
\usepackage[top=3cm, bottom=3cm, left=3cm, right=3cm]{geometry}

\begin{document}
\noindent
package1.souspackage2.souspackage3\\ 
\part*{\bfseries VecteurDeDouble}
\noindent
Author: TELECOM BRETAGNE\\
Version: 1.2\\
\\
\noindent
\begin{tabular}{|p{3cm}|p{12cm}|}
\hline
\multicolumn{2}{|l|}{\bfseries {Field Summary}} \\
\hline
private int[ ] &  \bfseries \ul{elements} \tabularnewline
  & {\ \ \ \ contient la valeur des \'el\'emnets du "VecteurDeDouble"}\\
\hline
private boolean[ ] &  \bfseries \ul{setsRealises} \\
  & {\ \ \ \ setsRealises[i]=true si l'element correspond du "VecteurDeDouble" a \'et\'e affect\'e (flase sinon)}\\
\hline
\end{tabular}
\\
\\
\\
\begin{tabular}{|p{15.4cm}|}
\hline
{\bfseries Constructor Summary}\\
\hline
{\bfseries \ \ \ul{VecteurDeDouble}} (\textit{int taille}) \\
\noindent
 \ \ \ \ \ Construit un "VecteurDeDouble" de taille \'el\'elements, non affect\'es. R\'ealise les instanciations et initialisations n\'ecessaires. \\
\hline
\end{tabular}
\\
\\
\\
\begin{tabular}{|p{3cm}|p{12cm}|}
\hline
\multicolumn{2}{|l|}{\bfseries Method Summary}\\
\hline
int & {\bfseries \ul{nbElements} }(\textit{})\tabularnewline

  & {\ \ \ \ Renvoie le nombre d'\'el\'ements du "VecteurDeDouble".}\\
\hline
void & {\bfseries \ul{setelement} }(\textit{ int indice, double valeur})\tabularnewline

  & { \ \ \ \ L'\'el\'ements  d'indice indice du "VecteurDeDouble" est affich\'e du double valeur pass\'e en param\`etre. Effectue, si n\'ecessaire, un agrandissement des tableaux elements et setsRealises, ainsi que leurs mises \`a jour.}\\
\hline
\end{tabular}
\\
\\
\\
\begin{tabular}{|p{15.4cm}|}
\hline
\Large{Field Summary Detail}\\
\hline
\end{tabular}
\\
\\
\begin{tabular}{p{1cm} l}
\multicolumn{2}{l}{private int[ ]  {\bfseries {elements}}}\\
& contient la valeur des \'el\'emnets du "VecteurDeDouble".
\end{tabular}
\\
\rule{\linewidth}{.4pt}
\\
\\
\begin{tabular}{p{1cm} l}
\multicolumn{2}{l}{private boolean[ ] {\bfseries {setsRealises}}}\\
& setsRealises[i]=true si l'element correspond du "VecteurDeDouble" a \'et\'e affect\'e (flase sinon).
\end{tabular}
\\
\\
\\
\begin{tabular}{|p{15.4cm}|}
\hline
\Large{Constructor Detail}\\
\hline
\end{tabular}
\\
\\
\begin{tabular}{p{1cm} p{1cm} l}
\multicolumn{3}{l}{{\bfseries VecteurDeDouble} (\textit{int taille})}\\
\\
& \multicolumn{2}{l}{Construit un "VecteurDeDouble" de taille \'el\'elements.}\\
\\
& \multicolumn{2}{l}{\textbf{Param�tres:}}\\
&&\texttt{variable} - Utilisation de la variable\\
& \multicolumn{2}{l}{\textbf{Returns:}}\\
&& Aucun retour \\
& \multicolumn{2}{l}{\textbf{Throws:}}\\
&&\texttt{NomException} - Explication de l'exception\\
\end{tabular}
\\
\\
\\
\noindent
\begin{tabular}{|p{15.4cm}|}
\hline
\Large{Method Detail}\\
\hline
\end{tabular}
\\
\\
\\
\noindent
\begin{tabular}{p{1cm} p{1cm} l}
\multicolumn{3}{l}{int {\bfseries nbElements} ()}\\
\\
& \multicolumn{2}{l}{Renvoie le nombre d'\'el\'ements du "VecteurDeDouble".}\\
\\
& \multicolumn{2}{l}{\textbf{Param�tres:}}\\
&&\texttt{compteur} - pour parcourir le tableau\\
& \multicolumn{2}{l}{\textbf{Returns:}}\\
&& le nombre d'�l�ments du "VecteurDuDouble"\\
& \multicolumn{2}{l}{\textbf{Throws:}}\\
&&\texttt{NomException} - Explication de l'exception\\
\end{tabular}
\\
\rule{\linewidth}{.4pt}
\\
\\
\\
\begin{tabular}{p{1cm} p{1cm} l}
\multicolumn{3}{l}{void {\bfseries {setelement} }(\textit{ int indice, double valeur})}\\
\\
& \multicolumn{2}{l}{Affiche L'\'el\'ements  d'indice indice du "VecteurDeDouble".}\\
\\
& \multicolumn{2}{l}{\textbf{Param�tres:}}\\
&&\texttt{variable} - Utilisation de la variable\\
& \multicolumn{2}{l}{\textbf{Returns:}}\\
&& Aucun retour \\
& \multicolumn{2}{l}{\textbf{Throws:}}\\
&&\texttt{NomException} - Explication de l'exception\\
\end{tabular}


\end{document}
